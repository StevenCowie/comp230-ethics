% Please do not change the document class
\documentclass{scrartcl}

% Please do not change these packages
\usepackage[hidelinks]{hyperref}
\usepackage[none]{hyphenat}
\usepackage{setspace}
\doublespace

% You may add additional packages here
\usepackage{amsmath}

% Please include a clear, concise, and descriptive title
\title{What are the effects hypersexualisation of female characters has on the games industry and what can be done to avoid this issue?}

% Please do not change the subtitle
\subtitle{COMP230 - Ethics and Professionalism Essay}

% Please put your student number in the author field
\author{1605240}

\begin{document}

\maketitle

\section{Abstract}
With the video games industry now being a hundred billion dollar industry \cite{bowey2017don} and constantly expanding, it is very important that females are portrayed in the correct way. People often stereotype gamers to be mostly young boys however a study showed that 48 percent of gamers are female \cite{heron2014sexism}. Hypersexualisation is the topic of discussion in this paper, to look at both sides of the argument for it and discuss solutions to make everyone happy.
\newpage

\section{Introduction}
With the video games industry now being a hundred billion dollar industry \cite{bowey2017don} and constantly expanding, it is very important that females are portrayed in the correct way. Ethical issues in the games industry aren't dealt with the best way\cite{flick2016ask}. People often stereotype gamers to be mostly young boys however a study showed that 48 percent of gamers are female \cite{heron2014sexism}. This introduces us to the main topic of discussion which is what are the effects hypersexualisation of female characters has on the games industry and what can be done to try and solve the issue, I will also be looking at the other side of the argument which is why developers should be allowed to hypersexualise and stereotype  females.

\section{What is hypersexualisation?}
Hypersexualisation in games is when characters, usually female are portrayed in such a sexual manner. They use many ways to portray this such as "unnaturally large breasts, a shapely behind, a narrow waist, blushing face, big red pouting lips and visible nipples \cite{waern2005hypersexual}.

\section{Issues with hypersexualisation}
With video games being useful for education\cite{lekka2014computer} and being used more in education\cite{jenson2007girls} girls are at a disadvantage if they do not play games\cite{jenson2007girls}. Although female gamers are on the rise, this has mainly been because of mobile games\cite{bbc}. This is an issue because the more games are used and the more females that don't know how to play them and won't take anything away from it, they will most likely get annoyed. A big reason less females are interested in games than males is down to hypersexualisation where and when they feel like they can't connect with the characters.\cite{desai2016effects} \newline


The main issue with hypersexualisation from my research is the fact that females feel like they're stereotyped too much \cite{heron2014sexism}, this can go on to lead to decreased sales as there would be less female customers\cite{stats} and links to my point above.  \newline


Another issue is the fact that children are exposed to characters with hypersexual features\cite{martin2011playing} in many games these days, this can cause them to see women in a more traditional gender role\cite{martin2011playing}, this has lead to quite a sexist games industry. Zoe Quinn goes onto talk about GamerGate\cite{gamergate}\cite{bowey2017don}, this talks about the treatment of women and their roles. This was good of her to speak up because it caused an out roar and for more attention to be drawn to the matter. \newline 


The other issue caused by hypersexualisation is it can ruin games and take away from the playing experience\cite{venturebeat}, as seen in games such as Mass Effect 2 when the camera pans at inappropriate angles, is there any real need to do this? I believe this just subtracts from what is going on in the game.


\section{Other side of the argument}
The other side of the argument is games are how developers express how they feel and it is their work of art. A quote from the developer of Mass Effect about a character who is considered to be hypersexualised, "That's part of her character design, she's the femme fatale. It's part of her character and the fact that she's beautiful and this beauty is part of what helps her. As you get to know her, you realize there's more to her."\cite{venturebeat}.  This is a valid point because everyone has the right to freedom of speech and to be creative in their own way. These games are usually targeted towards males but there are games that are more targeted towards females\cite{bowey2017don}\cite{waern2005hypersexual}. \newline


It's not just females that are sexualized in games, it is males too\cite{heron2014sexism}, they do this because males enjoy playing as fantasy characters who they can only imagine to be. Males aren't sexualized as bad as women are but they are buffed up and it still creates an unrealistic goal of what a person can be, so why do they not complain too?


Also certain females prefer the sexualized avatar in some cases\cite{waern2005hypersexual}, in this study females preferred the avatars who were clothed but at the same time had a sexualized body, this can be for the same reason as males because they want to play as a fantasy character.

\section{Solutions}
One potential solution is to hire a more diverse workers, especially women \cite{smith2016understanding} this would allow them to have more of a say and for the user to have characters that are better represented. This could lead to more sales for the developers as there would be more females playing it as females don't like only being able to use a male avatar\cite{heron2014sexism} and it could lead to a change of advertising strategy as games are mainly aimed at male gamers\cite{waern2005hypersexual}. Female workers only made up three percent on the workforce in 1989\cite{pbs}. \newline


Another solution albeit not a very practical one is to make different games for different genders, studies have shown that men and women prefer different gaming experiences\cite{desai2016effects}. Games targeted specifically for females could tailored to be more in depth for them as females have an easier time recognizing the emotion on NPCs\cite{desai2016effects}. This could increase the amount of females who play games which in turn makes the industry more diverse. \newline


Education should play a large role in ethics in games\cite{lekka2014computer}, such that an ethical list\cite{scavarelli2014cindr} should be put together so developers all know what is ethical to do and they don't just go and add a hypersexual character with no in game context.

\section{Conclusion}
To conclude my findings point to toning hypersexualisation down in video games and would lead to many more benefits to the games industry. The biggest benefit would be an increase of female gamers and workers and it would hopefully lead to less sexism in the industry. To help tone it down education should be used to educate developers on ethics. If they feel the need to add a hypersexualised character then they should only be there if there are adding something to the game. Other solutions could be effective too but this is the most cost and time effective.

\bibliographystyle{IEEEtran}
\bibliography{references}

\end{document}
